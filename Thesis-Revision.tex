Mudar o título

Incorporar o Capítulo 2 de Metodologia na Introdução

Trabalhos da Tabela 5 são apenas trabalhos acadêmicos.
Trabalhos da Tabela 6 são ferramentas complementares.

Deixar claro de que formas os trabalhos da Tabela 5 contribuíram para o estudo.

ERtext Tool > ERtext Proposal

Figura 9 explicar melhorar o atributo target

Explicar melhor o target dos RelationSideLeft and RelationSideRight

Deixar explícito o não suporte ao relacionamento n'ário

Geração da versão gráfica é estática

Incluir os trabalhos relacionados comentados pelo Ronaldo
Dois trabalhos da IEEE
1) CNLER: A Controlled Natural Language for Specifying and Verbalising Entity Relationship Models - https://aclanthology.org/U19-1017.pdf
2) An entity-relationship programming language - https://ieeexplore.ieee.org/document/31369
3) A graph grammar for entity relationship diagrams - https://ieeexplore.ieee.org/document/7819271

Detalhamento no texto do cálculo (pontuação) da randomização
Explicar mais detalhadamente o MI e o RI
Explicar melhor como foi realizada a avaliação qualitativa dos modelos produzidos.

Incluir nas ameaças a questão do momento e o conhecimento prévio dos alunos.
Incluir como ameaça a indicação e o controle do tempo destinado para realizar cada uma das tarefas.

Tabela 10 destaca a similaridade dos desvios padrões entre as duas abordagens.

Figura 41 comentar sobre os Outliers, talvez exibir o gráfico com ou sem os Outliers.

Processo semiautomático na transformação entre os modelos
Adicionar como ideia de trabalho futuro permitir aos usuários determinarem os ajustes.
