------------------------------------------------------------------------------------
PROF. RONALDO
------------------------------------------------------------------------------------

[-] Mudar o título
[-] Incorporar o Capítulo 2 de Metodologia na Introdução
[X] Trabalhos da Tabela 5 são apenas trabalhos acadêmicos.
[X] Trabalhos da Tabela 6 são ferramentas complementares.
[X] Deixar claro de que formas os trabalhos da Tabela 5 contribuíram para o estudo.
[X] ERtext Tool > ERtext Proposal
[X] Figura 9 explicar melhorar o atributo target
[X] Explicar melhor o target dos RelationSideLeft and RelationSideRight
[X] Deixar explícito o não suporte ao relacionamento n'ário
[X] Geração da versão gráfica é estática
[-] Incluir os trabalhos relacionados comentados pelo Ronaldo
	Dois trabalhos da IEEE
	1) CNLER: A Controlled Natural Language for Specifying and Verbalising Entity Relationship Models - https://aclanthology.org/U19-1017.pdf 
			(https://gojs.net/latest/samples/entityRelationship.html)
	2) An entity-relationship programming language - https://ieeexplore.ieee.org/document/31369 ()
	3) A graph grammar for entity relationship diagrams - https://ieeexplore.ieee.org/document/7819271

------------------------------------------------------------------------------------
PROF. JULIANO
------------------------------------------------------------------------------------

[X] Detalhamento no texto do cálculo (pontuação) da randomização
[X] Explicar mais detalhadamente o MI e o RI
[X] Explicar melhor como foi realizada a avaliação qualitativa dos modelos produzidos.
[X] Incluir nas ameaças a questão do momento e o conhecimento prévio dos alunos.
[X] Incluir como ameaça a indicação e o controle do tempo destinado para realizar cada uma das tarefas.
[X] Tabela 10 destaca a similaridade dos desvios padrões entre as duas abordagens.
[X] Figura 41 comentar sobre os Outliers, talvez exibir o gráfico com ou sem os Outliers.
[X] Processo semiautomático na transformação entre os modelos, adicionar como ideia de trabalho futuro permitir aos usuários determinarem os ajustes.

------------------------------------------------------------------------------------
PROF. BASSO
------------------------------------------------------------------------------------

[X] Arrumar quebra de texto (paragrafos) nas figuras dos Emocards
[X] Citar trabalho da QAnubis 


