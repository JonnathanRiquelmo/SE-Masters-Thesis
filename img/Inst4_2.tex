\pgfplotsset{testbar/.style={
            xbar stacked,
            legend cell align=left,
            legend style={
                legend columns=6,
                font=\scriptsize,
                at={(xticklabel cs:1.0)},
                anchor=north east,
                draw=none
                },
            width=10cm,
            axis y line*= none, 
            axis x line*= bottom,
            xmajorgrids = false,
            xmin=0,xmax=25,
            ytick = data,
            yticklabels = {
            {\scriptsize Entity}, 
            {\scriptsize Referential Attribute},
            {\scriptsize Descriptive Attribute},
            {\scriptsize Binary Relationship},
            {\scriptsize Ternary Relationship}, 
            {\scriptsize Self-relationship},
            {\scriptsize Cardinality},
            {\scriptsize Generalization}
            },
            tick align = outside, 
            xticklabel style = {font=\scriptsize},
            xtick pos = left,
             bar width=5mm, 
             y=7mm,
             enlarge y limits={abs=0.450},% 0.5 + 0.5*(y - bar width)/y [TeX.sx #47995] #47995]
            nodes near coords,
            nodes near coords align=center,%Move values in bar
            every node near coord/.append style={
                black,
                font=\scriptsize,
                text opacity=1,
                fill=white,
                fill opacity=0.5,
                outer sep=\pgflinewidth
            }
        }}
    \begin{tikzpicture}
    \begin{axis}[testbar] 
    \addplot[pattern color=red,pattern=north east lines] coordinates
        {(0,8)(0,7)(0,6)(0,5)(0,4)(0,3)(0,2)(0,1)};
    \addplot[pattern color=teal,pattern=vertical lines] coordinates
        {(0,8)(0,7)(0,6)(1,5)(2,4)(1,3)(0,2)(0,1)};   
    \addplot[pattern color=gray, pattern=grid] coordinates
        {(0,8)(0,7)(2,6)(2,5)(5,4)(0,3)(0,2)(3,1)};   
    \addplot[pattern color=magenta, pattern=north west lines] coordinates
        {(3,8)(4,7)(2,6)(0,5)(6,4)(3,3)(0,2)(3,1)};   
    \addplot[pattern color=blue, pattern=horizontal lines] coordinates
        {(4,8)(4,7)(6,6)(11,5)(7,4)(5,3)(5,2)(6,1)}; 
    \addplot[pattern color=green, pattern=crosshatch dots] coordinates
        {(18,8)(17,7)(15,6)(11,5)(5,4)(16,3)(20,2)(13,1)};
    \legend{1-Disagree, 2, 3, 4, 5, 6-Agree}
    \end{axis}
    \end{tikzpicture}

% \begin{figure}[!ht]
% \centering
% \caption{Resultados do formulários de avaliação.}
% \begin{tikzpicture}
% \begin{axis}[
%     xbar stacked,
%     legend cell align=center,
%     legend style={
%     legend columns=5,
%         at={(xticklabel cs:1.0)},
%         anchor=north east,
%         draw=none
%     },
%     ytick=data,
%     axis y line*=none,
%     axis x line*=bottom,
%     tick label style={font=\small},
%     legend style={font=\small},
%     label style={font=\small},
%     xtick={0,5,10},
%     xticklabel= {},
%     bar width=7mm,
%     ylabel={Formulário/Grupo},
%     yticklabels={F1-C, F1-E, F2-C, F2-E, F3-C, F3-E},
%     xmin=0,
%     xmax=10,
%     area legend,
%     y=9mm,
%     enlarge y limits={abs=0.825},
%     nodes near coords,
%     nodes near coords align=center,
%     every node near coord/.append style={
%         black,
%         font=\small,
%         text opacity=.65,
%         fill=white,
%         fill opacity=0.6,
%         outer sep=\pgflinewidth
%     }
% ]
% %NOTA 1
% \addplot[pattern color=red,pattern=horizontal lines] coordinates
% {(0,6)(0,5)(0,4)(0,3)(0,2)(0,1)};
% %NOTA 2
% \addplot[pattern color=orange,pattern=grid] coordinates
% {(1,6)(1,5)(1,4)(0,3)(1,2)(0,1)};   
% \addplot[pattern color = green, pattern=crosshatch dots] coordinates
% % NOTA 3
% {(3,6)(3,5)(5,4)(2,3)(5,2)(1,1)};   
% \addplot[pattern color=blue, pattern =vertical lines ] coordinates
% %NOTA 4
% {(2,6)(3,5)(1,4)(4,3)(3,2)(3,1)};   
% \addplot[pattern color=gray, pattern = dots] coordinates
% %NOTA 5
% {(3,6)(3,5)(2,4)(4,3)(0,2)(6,1)};   
% \legend{1-Disagree, 2, 3, 4, 5-Agree}

% \end{axis}
% \end{tikzpicture}
% \footnotesize
% \label{img:respostas1}
% 	\fonte{O autor.}
% \end{figure}