\begin{resumo}[Abstract]
%=============================================
%================== CONTEXT ==================
%=============================================
With the advance of technology, databases have become vital elements in modern society.
Databases are sets of data stored to make sense of a specific domain.
The stored information is considered a highly relevant asset in contemporary organizations.
Thus, the effective use of databases is deeply relevant for the maintenance and correct continuation of their activities.
%=============================================
%================== PROBLEM ==================
%=============================================
Once that is said, training in this area for professionals from academia must be constant, and it is a fundamental point to which higher education institutions must pay special attention.
However, the variety of database systems technologies that have become available in recent years, with the vast majority focused on graphical approaches, makes it hard to choose tools for entity-relationship (ER) modeling in the industry and, consequently, in academia.
%=============================================
%============ SOLUTION/CONTRIBUTION ==========
%=============================================
Hence, aiming to contribute with a relevant open-source alternative, this study proposes the continuation of a modeling language project.
The goal is to implement a tool based on a textual domain-specific language (DSL) to support the teaching-learning process of conceptual database modeling.
% The use of DSLs provide ways to specify and model domains in a faster and more productive way, as they are languages with expressiveness limited to particular domains, differentiating them from general-purpose languages.
DSLs allow specifying and modeling domains in a faster and more productive way, as they are expressiveness limited to particular domains, which do the general-purpose languages been differentiating from them.
%==========================================================================
%================== STATE-OF-THE-ART and PROPOSED METHOD ================== 
%==========================================================================
In this sense, we carried out a state-of-the-art investigation on database model transformation mechanisms.
This survey that we have carried out uses a systematic literature mapping protocol.
The execution of this protocol comprises a final set of 12 primary studies that report the propositions of DSLs and the mechanisms they use.
Then, there was the selection of the Xtext framework to support the development of the tool.
After establishing a prototype of the proposed tool, we planned and carried out a replication of a preliminary empirical evaluation with a group consisting of thirty-three (33) participants.
With the feedback received in the process, we reached a framework of knowledge that supports the evolution of the proposal in a stable version of the tool, called ERtext.
To evaluate this stable version, an empirical evaluation similar to the preliminary one will be conducted.
% The intention will be to verify the effort (time), accuracy, recall, F-Measure, perceived usefulness, and ease of use of the tool with a textual approach developed compared to a tool with a graphical approach.
% In addition, it is also intended to evaluate the artifacts that can be generated, as well as the resources added later, and that were not evaluated before.
The intention will be to verify the effort (time), accuracy, recall, F-Measure, perceived usefulness, and ease of use of both approaches a textual based on our proposal tool compared to a graphical-based on an academic tool.
Moreover, we intend to evaluate the artifacts generated and also the new features added not assessed yet.
%=============================================
%================== RESULTS ==================
%=============================================

\vspace{\onelineskip}

 \noindent 
 \textbf{Key-words}: Database. Database Design and Modeling. Conceptual Modeling. Domain Specific Language.
\end{resumo}