% -----------------------------------------------
% Folha de aprovação [OBRIGATÓRIO]
% -----------------------------------------------
% Este é um exemplo de Folha de aprovação, elemento obrigatório da NBR 14724/2011 (seção 4.2.1.3).
% Você pode utilizar este modelo até a aprovação do trabalho.
% Após isso, altere o conteúdo deste arquivo para inserir uma imagem da página assinada pela banca usando
% o modelo que está no final deste arquivo.

% -----------------------------------------------
% Folha de aprovação antes da defesa do TCC
% -----------------------------------------------
%\begin{comment}
\begin{folhadeaprovacao}
  \begin{center}
    {\ABNTEXchapterfont\large\imprimirautor}

    \vspace*{\fill}%\vspace*{\fill}
    {\ABNTEXchapterfont\bfseries\Large\imprimirtitulo}
    \vspace*{\fill}

    \hspace{.45\textwidth}
    \begin{minipage}{.5\textwidth}
        \imprimirpreambulo
    \end{minipage}%
    \vspace*{\fill}
  \end{center}

  \begin{center}
    \imprimirtipotrabalho held and approved in April 29, 2022.

    Board of examiners:
  \end{center}

  \assinatura{\textbf{\imprimirorientador} \\ Supervisor \\ Unipampa}
  \makeatletter
  \abntex@ifnotempty{\imprimircoorientador}{%
    \assinatura{\textbf{\imprimircoorientador} \\ Co-supervisor \\ Unipampa}%
  }
  \makeatother
  \assinatura{\textbf{Prof. PhD. Juliano Varella de Carvalho} \\ Feevale}
  \assinatura{\textbf{Prof. PhD. Ronaldo dos Santos Mello} \\ UFSC}

\end{folhadeaprovacao}
%\end{comment}
% -----------------------------------------------
% Folha de aprovação após a defesa do TCC com a imagem da folha de aprovação assinada pela banca.
% -----------------------------------------------
\begin{comment}
\begin{folhadeaprovacao}

% Escolher entre uma das seguintes opções para inclusão da folha de aprovação
% Versão assinada em arquivo PDF (incluir no arquivo principal o comando \usepackage{pdfpages})
%\includepdf{pretextuais/aprovacao.pdf}

% Ou, versão assinada em arquivo de imagem (jpg, png, etc)
% Mas prefira em PDF. Em imagem é preciso acertar os recuos das margens:
%\vspace*{-4cm}
%\hspace*{-3.5cm}
%\includegraphics[width=\paperwidth]{pretextuais/aprovacao}

\end{folhadeaprovacao}
\end{comment}
