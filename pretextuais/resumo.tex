\begin{resumo}[Resumo]
% ==============================================
% ================== CONTEXTO ==================
% ==============================================
Os bancos de dados tornaram-se elementos vitais na sociedade moderna com o avanço da tecnologia.
Bancos de dados são conjuntos de dados armazenados para dar sentido a um domínio específico.
A informação armazenada é considerada um ativo de grande relevância nas organizações contemporâneas.
% ==============================================
% ================= PROBLEMA ===================
% ==============================================
Como a qualificação profissional exige a assimilação de conceitos fundamentais, incluindo modelagem conceitual, lógica e física de banco de dados, alternativas de ferramentas podem ser exploradas nas disciplinas de projeto e modelagem de banco de dados.
As instituições de ensino superior devem prestar atenção às abordagens que melhor mapeiam as necessidades dos alunos de acordo com suas preferências para facilitar o aprendizado.
No entanto, a variedade de tecnologias que se tornaram disponíveis nos últimos anos, com a grande maioria focada em abordagens gráficas, dificulta a escolha de ferramentas para modelagem entidade-relacionamento (ER) na indústria e, consequentemente, na academia.
%=============================================
%============ SOLUÇÃO/CONTRIBUIÇÃO ==========
%=============================================
Assim, visando contribuir com uma alternativa relevante de código aberto, este estudo propôs dar continuidade a um projeto de linguagem de modelagem.
O objetivo foi implementar uma ferramenta baseada em uma linguagem textual de domínio específico (DSL) para apoiar o processo de ensino-aprendizagem de modelagem conceitual de banco de dados.
As DSLs permitem especificar e modelar domínios de forma mais rápida e produtiva. Elas se diferenciam das linguagens de uso geral pois sua expressividade é limitada para domínios específicos.
% ================================================ =========================
% ================= ESTADO DA ARTE E MÉTODO PROPOSTO =======================
% ================================================ =========================
Nesse sentido, realizamos uma investigação do estado da arte sobre os mecanismos de transformação de modelos de banco de dados.
Esta investigação utilizou um protocolo de mapeamento sistemático da literatura, e compreendeu um conjunto final de 12 estudos primários que relatam as proposições das DSLs e os mecanismos que utilizam.
Após estabelecer um protótipo da proposta, planejamos e realizamos uma avaliação empírica preliminar com um grupo composto por trinta e três (33) participantes.
Com o feedback recebido no processo, chegamos a um arcabouço de conhecimento que suporta a evolução da proposta em uma versão estável da ferramenta, chamada ERtext.
Assim, realizamos mais duas avaliações empíricas (25 e 15 participantes, respectivamente) semelhantes à avaliação preliminar.
A intenção foi verificar o esforço (tempo), acurácia, \textit{recall}, medida F, utilidade percebida e facilidade de uso de ambas as abordagens: uma ferramenta com base em texto (ERtext) comparada à uma ferramenta gráfica (brModelo).
Além disso, avaliamos os artefatos gerados, bem como as novas funcionalidades adicionadas que ainda não haviam sido avaliadas.
% ================================================
% ================= RESULTADOS ===================
% ================================================
No final, os três experimentos controlados combinados tiveram setenta e três (73) participantes.
Realizamos testes estatísticos detalhados de hipótese, bem como análises qualitativas.
Em conclusão, no geral os resultados indicaram que não existem diferenças significativas \textit{de facto} entre as abordagens, demonstrando assim que nossa proposta textual é viável por equiparar-se à já reconhecida e amplamente utilizada abordagem gráfica .

\vspace{\onelineskip}

 \noindent
 \textbf{Palavras-chave}: Projeto e Modelagem de Banco de Dados. Modelagem Conceitual. Linguagem Específica de Domínio.
\end{resumo}