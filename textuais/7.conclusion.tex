%=========================================================
\chapter{Final Remarks}
\label{chap:conclusion}
%=========================================================

% Esta dissertação se propôs até o momento a realizar um estudo que aborda a área de projeto e modelagem conceitual de \acp{db} relacionais.
% Para tanto, foi necessário realizar um desenho de pesquisa que orientou toda a execução das atividades necessárias para atingir os objetivos específicos.
% O projeto envolve ainda uma pesquisa bibliográfica para o embasamento teórico, a condução de um mapeamento sistemático, encontrar estudos relacionados, o desenvolvimento da solução proposta e ao menos duas avaliações empíricas para sua avaliação.
This master's thesis has proposed to execute a study that addresses the area of design and conceptual modeling of relational \acp{db}.
Therefore, it was necessary to carry out a research design that guided all the activities necessary to achieve the specific objectives.
The project also involved a bibliographical research for the theoretical basis, conducting a systematic mapping, finding related studies, developing the proposed solution and three (3) empirical evaluations for its evaluation.

% Sabemos que o conjunto de ferramentas que apoiam o projeto e modelagem de bancos de dados relacionais é relativamente amplo.
% Não esperamos reinventar a roda ou apenas aumentar essa fragmentação composta por diversas soluções em uma área já madura. 
% Em geral, este trabalho trata de uma experimentação.
% Estamos procurando verificar suposições e explorando uma alternativa em relação a abordagem mais utilizada neste meio, e portanto de certa forma desafiando o \textit{status quo} estabelecido.
% Como resultado esperamos contribuir com uma ferramenta que seja potencialmente útil para o processo de ensino-aprendizagem.
We know that the set of tools that support the design and modeling of relational \acp{db} is relatively wide.
We do not expect to reinvent the wheel or just increase this fragmentation made up of several solutions in an already mature area.
In general, this work is about an experimentation.
We are looking to verify assumptions and exploring an alternative to the most used approach in this area, and therefore somewhat challenging the established \textit{status quo}.
As a result we hope to contribute with a tool that is potentially useful for the teaching-learning process.

% Nessa tentativa de expandir os limites e explorar um caminho alternativo, que acreditamos que deva ser encorajado dentro da academia, nós propomos uma \ac{dsl} e implementamos uma ferramenta completa de modelagem.
% Isso só foi possível através do framework Xtext, um conhecido \ac{lw} para criação de linguagens de abordagem textual.
In this attempt to push the boundaries and explore an alternative path, which we believe should be encouraged within academia, we propose a \ac{dsl} and implement a complete modeling tool.
This was only possible through the Xtext framework, a known \ac{lw} for creating textual approach languages.

% Executamos a replicação de um experimento empírico com um conjunto de 33 sujeitos para verificar a viabilidade da proposta e os resultados preliminares indicam que pode haver relevância no projeto construído em nosso estudo.
% Estamos cientes que mesmo que os resultados, após o último experimento planejado a ser realizado, mostrem em última análise que não há diferença significativa da nossa proposta para as ferramentas já existente, ainda assim o nosso conjunto coletivo de conhecimento permanecerá relevante no que diz respeito ao desenvolvimento e evolução de linguagens de modelagem para domínios específicos.
We performed the replication of an empirical experiment with a set of 33 subjects to verify the feasibility of the proposal and the preliminary results indicate that there may be relevance in the project built in our study.
% We are aware that even if the results, after the last planned experiment to be carried out, ultimately show that there is no significant difference from our proposal for the existing tools, still our collective body of knowledge will remain relevant with regard to development and evolution of modeling languages for specific domains.
We are aware that even if the results ended up ultimately showing that, after the last planned experiment, there was no significant difference from our proposal for the existing tools, still our collective body of knowledge will remain relevant with regard to development and evolution of modeling languages for specific domains.

% Isto se confirmou na execução dos outros dois (2) experimentos empíricos.
% Ao final, os três (3) experimentos combinados obtiveram setenta e três (73) participações.
% Conseguimos então avaliar de forma quantitativa e qualitativa uma série suposições.
% Nossa vontade principal era avaliar se nossa abordagem textual poderia ter um desempenho e aceitação de usuário tão efetiva, ou ao menos semelhante, a abordagem gráfica.
% Chegamos a conclusão que ambas as são similares, sem uma diferença estatística ou de qualidade de uso que se mostre verdadeiramente significativa dentre o conjunto de sujeitos participantes dos experimentos controlados. 
We confirmed this in the other two (2) empirical experiment executions.
%This was confirmed in the execution of the other two (2) empirical experiments.
In the end, the three (3) combined experiments had seventy-three (73) subjects.
We were then able to quantitatively and qualitatively evaluate a series of assumptions.
Our main goal was to assess whether our textual approach could perform the user acceptance as effectively or at least very close to the graphical one.
We concluded that both approaches are similar, without a statistical difference or quality of use that proves significant among the subjects participating in the controlled experiments.
%We came to the conclusion that both are similar, without a statistical difference or quality of use that proves to be truly significant among the set of subjects participating in controlled experiments.

% Temos ainda outras expectativas em relação a ferramenta desenvolvida que vão além de verificar se é uma alternativa viável  e promover colaboração, em razão de ser código aberto, bem como atender às necessidades dos usuários.
% Isto diz respeito a nossa vontade de avaliar se a abordagem textual pode ser de fato tão efetiva quanto a abordagem gráfica. 
% Pretendemos realizar uma melhor análise a partir de uma possível futura inclusão da solução nas atividades realizadas durante a condução de uma classe de banco de dados, assim performando as atividades continuamente da mesma forma com que ferramentas gráficas são utilizadas.
% !!!! We also have other expectations for the developed tool, which goes beyond checking whether it is a viable alternative and promoting collaboration since it is open-source, as far as to meet the users' needs.
% We also have other expectations to the developed tool that goes beyond checking if it is a viable alternative and promoting collaboration, since it is open-source, as well as meeting the needs of users.
% !!!! Hence, this concerns our willingness to assess whether the textual approach can be as efficacious as the graphical approach.
% We intend to carry out a better analysis based on a possible future inclusion of the solution in the activities performed while conducting a \ac{db} lessons, thus performing the activities continuously in the same way that graphical tools are used.
We still intend to conduct a better analysis based on the possible inclusion of the future solution in the activities carried out while teaching database lessons, thus performing the activities continuously in the same way used in the graphical tools.

% Desta forma será possível reverificar algumas vantagens que observamos inicialmente (Seção \ref{sec:intro_rationale}), e ainda avaliar se outras vantagens não descritas anteriormente podem ser adicionadas neste conjunto de benefícios.
% Entre as suposições iniciais que temos estão a de que a abordagem textual pode trazer grandes vantagem na velocidade de modelagem quando os usuários estão mais acostumados com a linguagem,  pode promever uma integração facilitada com outras linguagens, oferece qualidade superior de formatação, melhor controle de versão e indepêndencia de plataforma (no que diz respeito a sistemas operacionais).

% In this way, it will be possible to recheck some advantages that we observed initially (Section \ref{sec:intro_rationale}), and also to assess whether other advantages not previously described can be added to this set of benefits.
In this way, as we initially observed (Section \ref{sec:intro_rationale}), it will be possible to recheck some advantages and assess whether others not previously described could be added to this benefits set.
% Among the initial assumptions we have are that the textual approach can bring great advantages in modeling speed when users are more familiar with the language, can provide easier integration with other languages, offer superior formatting quality, better version control and platform independence (concerning operating systems).
Among the initial assumptions, we can assert that the textual approach can: bring good advantages in modeling speed when users are more familiar with the language; provide easier integration with other languages; offer superior formatting quality; better version control, and; platform independence (concerning operating systems).

% Por que usar Desktop e não Web? Incluir e responder esse questionamento. (JULIANO)
% Sabemos que a solução proposta utiliza uma perspectiva de plugin no momento.
% Isso se dá pela necessidade da construção ágil de uma infraestrutura completa que contemple desde parser, linker, typechecker até mesmo o compilador completo da linguagem criada.
% Não existe na atualidade frameworks como o Xtext que disponibilizem isso diretamente, de forma especializada, em um contexto web.
% Contudo, apesar da limitação imposta por um plugin Eclipse, i.e. funciona apenas localmente, o Xtext fornece suporte completo a LSP (Language Server Protocol), o que abre a possibilidade concreta da migração futura da nossa solução para uma plataforma web, capaz de ser executada diretamente em um navegador. 
We know that the proposed solution uses a plugin perspective at the moment.
This is due to the need for the agile construction of a complete infrastructure that includes from parser, linker, and type-checker to the compiler of the created language.
There is currently no Xtext-like framework that makes this available directly in a specialized web context.
However, despite the limitation imposed by an Eclipse plugin,\textit{ i.e.} works only locally, Xtext provides full \ac{lsp} support, which opens up the real possibility of the future migration of our solution to a web platform able to run directly in a browser.

%--------------------------------------------------------------------
\section{Publications}
%--------------------------------------------------------------------

% Além disso, destacamos que durante o período de realização deste estudo os resultados parciais foram submetidos a diversos eventos visando uma avaliação de nossos pares acerca do nosso trabalho.
Furthermore, we emphasize that during the period study, we have submitted the partial results to several events aiming at evaluations of expert peers about our work.
As follows, we list some publications that originated in the context of our research.
% Furthermore, we emphasize that during the period of this study, the partial results were submitted to several events aiming at an evaluation of our peers about our work.
% Below we list some publications that originated in the context of our research.
\linebreak 
\linebreak
\textbf{Published} 
    \begin{enumerate}[label=\roman*.]
        \item Lopes, J., Bernardino, M., Basso, F., Rodrigues. Avaliação preliminar de uma linguagem para a representação textual de modelos conceituais de bancos de dados. In \textit{Anais da IV Escola Regional de Engenharia de Software.} Porto Alegre, RS, Brasil: SBC, 2020. p. 306–-315.~\cite{eres}. 
        \item Lopes, J., Bernardino, M., Basso, F., Rodrigues. Multivocal Literature Mapping on DSLs and Tools for Entity-Relationship Modeling. In \textit{Anais da V Escola Regional de Engenharia de Software.} Porto Alegre, RS, Brasil: SBC, 2021.~\cite{eres:2021}
        \item Lopes, J., Bernardino, M., Basso, F., Rodrigues, E. Textual approach for designing database conceptual models: A focus group. In \textit{Proceedings of the 9th International Conference on Model-Driven Engineering and Software Development - MODELSWARD}: SciTePress, 2021. p. 171–-178.~\cite{modelsward21}. 
        \item Lopes, J., Bernardino, M., Basso, F., Rodrigues, E. Empirical evaluation of a textual approach to database design in a modeling tool. In \textit{Proceedings of the 23th International Conference on Enterprise Information Systems (ICEIS)}: SciTePress, 2021. p. 208--215.~\cite{iceis21}. 
        \item Lopes, J., Bernardino, M., Basso, F., Rodrigues, E. Textual-based DSL for Conceptual Database Modeling: A Controlled Experiment. In \textit{Anais do XXXVI Simpósio Brasileiro de Bancos de Dados (SBBD)}: SBC, 2021.~\cite{sbbd:2021}
        \item Lopes, J., Bernardino, M., Basso, F., Rodrigues, E. Entity-Relationship Modeling Tools and DSLs: is it still possible to advance the state of the art from observations in practice? In \textit{Proceedings of the 24th International Conference on Enterprise Information Systems (ICEIS)}: SciTePress, 2022. p. 8.~\cite{iceis:2022}.
    \end{enumerate} 

% \linebreak
\\~\\
% \textbf{Accepted to be Published} 

%     \begin{enumerate}[label=\roman*.]
%         \item Lopes, J., Bernardino, M., Basso, F., Rodrigues, E. Entity-Relationship Modeling Tools and DSLs: is it still possible to advance the state of the art from observations in practice? In \textit{Proceedings of the 24th International Conference on Enterprise Information Systems (ICEIS)}. [S.l.]: SciTePress, 2022.
%     \end{enumerate} 

% \linebreak
\\~\\
\textbf{Under Review}  

\begin{enumerate}[label=\roman*.]
    \item Journal of Information Data Management (JIDM), 2022.
\end{enumerate} 

\linebreak
\\~\\
\textbf{To be Submitted} 
    \begin{enumerate}[label=\roman*.]
        \item We intend to write a complete paper and submit it to \textbf{Software and System Modeling (SoSyM)}.
        The general idea will condense all previously published works reporting the research story and comparing the results, including the last controlled experiment.
        % The general idea will be to condense all previous works already published, reporting the research story and comparing the results including the last controlled experiment.
    \end{enumerate} 
% Resgatando trabalhos prévios publicados, contando história da pesquisa, ao mesmo tempo mencionando os resultados do ultimo experimento em comparação com os experimentos anteriores.

%--------------------------------------------------------------------
\section{Future Work}
%--------------------------------------------------------------------

% Existem outras melhorias que estamos desenvolvendo. 
% Pretendemos implementá-las com a continuidade do projeto, tornando-as públicas no repositório do projeto.
% Estes recursos buscam melhorar a experiência de uso e aumentar o espectro de artefatos que podem ser gerados.
% Entre os principais recursos que pretendemos adcionar, estão:
There are other improvements that we are developing.
We intend to implement them as the project continues, making them public in the repository.
These resources seek to improve the user experience and increase the spectrum of artifacts generated by the solution.
%These resources seek to improve the user experience and increase the spectrum of artifacts that can be generated.
Among the main features we intend to add are:

% \begin{itemize}
%     \item mais templates proposals, especialmente para modelos completos (atualmente existe apenas um disponível);
%     \item melhorar a validação com base em escopo, em especial para entidades ternárias (atualmente esta validação se dá de forma parcial);
%     \item criar geradores de DDLs para procedimentos armazenados cobrindo operações CRUD;
%     \item continuar a implementação do suporte de mineração de código\footnote {Code Mining: \url{https://www.eclipse.org/eclipse/news/4.8/platform_isv.php}} usando os recursos lançados no projeto Eclipse Photon (atualmente esta funcionalidade está parcialmente implementada, mas carece de mais testes de software para ser disponibilizada no plugin).
%     O objetivo será mostrar resumos ao usuário do número de entidades, incluindo aquelas derivadas de relacionamentos N:N, bem como o total de relacionamentos.
%     \item criar mais geradores \ac{sql}, aumentando assim o conjunto de plataformas que podem ser alvo (atualmente existe suporte a PostgreSQL e MySQL).
%     \item implementar meios de personalização de coloração dos modelos conceitual (diagrama com a PlantUML), lógico (.html) e físico (\ac{sql}) gerados automaticamente. Atualmente a coloração é definida estaticamente nos geradores desenvolvidos, mas essa opção é algo que percebido como sugestão direta dos sujeitos dos experimentos.
%     \item Disponibilizar a geração automática de diagramas de ocorrência para relacionamentos indicados na gramática.
%     \item A geração de diagramas dinâmicos, possibilitando a modelagem de forma gráfica que gere nossa \ac{dsl} automaticamente, tornando assim nossa solução uma ferramenta de modelagem bidirecional.
%     Atualmente temos como decisão de projeto utilizar o Sprotty\footnote{\url{https://github.com/eclipse/sprotty}} como framework para diagramação, substituindo a PlantUML Nos projetos relacionados que pretendemos dar continuidade. Isso acontece pelo Sprotty ter suporte total ao Xtext, possibilitando seu uso como backend (com requisições via Language Server Protocol - LSP) e Typescript  (um superset do Javascript com suporte a tipagem estática e OO) no frontend.
    % \item Implementar técnicas de \ac{rte}, ou seja, métodos de engenharia reversa buscando maneiras de gerar os artefatos (modelos conceituais, lógicos e físicos) a partir de instruções \ac{sql}.
    % \item Investigar uma possível integrações com frameworks \ac{orm} e linguagens de programação compatíveis.
% \end{itemize}

\begin{enumerate} [label=\roman*.]
    \item more \textbf{templates proposals}, especially for full templates (currently, there is only one available);
    \item \textbf{improve scope-based validation}, especially for ternary entities (currently, this validation is partial);
    \item create DDL \textbf{generators} for \textbf{stored procedures} covering CRUD operations;
    \item carry on implementing  \textbf{code mining support}\footnote {Code Mining: \url{https://www.eclipse.org/eclipse/news/4.8/platform_isv.php}} using features released in the Eclipse Photon project (currently, this functionality is partially implemented but needs more software testing to be made available in the plugin).
    The objective will be to show the user summaries of the number of entities, including those derived from N:N relationships, as well as the total of relationships;
    \item create \textbf{more \ac{sql} generators}, thus increasing the set of platforms that can target (currently, we support PostgreSQL and MySQL);
    \item implement \textbf{customized coloring features} of the automatically generated conceptual (diagram with PlantUML), logical (.html), and physical (\ac{sql}) models. 
    Currently, the coloring is defined statically in the developed generators, but we perceived that this option is something as a direct suggestion of the subjects of the experiments;
    % Currently, the coloring is defined statically in the developed generators, but this option is something that is perceived as a direct suggestion of the subjects of the experiments;
    \item provide automatic generation of \textbf{occurrence diagrams} for indicated relationships in the models;
    \item the generation of \textbf{dynamic diagrams} enables the modeling in a graphical way that generates our \ac{dsl} automatically, thus making our solution a \textbf{bidirectional modeling tool}. We currently have the design decision to use Sprotty\footnote{Sprotty Framework: \url{https://github.com/eclipse/sprotty}} as a diagramming framework, replacing PlantUML in related projects that we intend to continue. It happens because Sprotty has full support for Xtext, allowing its use as a backend (with requests via \ac{lsp}) and Typescript (a Javascript superset with support for static typing and OO) on the frontend;
    \item implement \ac{rte} techniques, \textit{i.e.}, \textbf{reverse engineering} methods looking to generate artifacts (conceptual, logical, and physical models) from \ac{sql} instructions;
    \item investigate the possible \textbf{integration} with \textbf{\ac{orm} frameworks} and compatible programming languages;
    \item Provide a semi-automatic process for transforming between models, allowing users to determine adjustments;
    \item Generators for NoSQL database models.
\end{enumerate}



% IDEIAS QUE SURGIRAM NA REUNIÃO DE 06/04/2022

% Colocar como trabalho futuro
% Ideia de fazer a enganharia reversa, recebendo um SQL para gerar a DSL (passando pelo lógico e diagrama conceitual) - Round-tripping engineering!
% parser a partir de um SQL, ou conexão direta com o banco de dados e partindo desse usuário (eg dbAdmin) com acesso a tabelas de sistema (de repente puxando uma AST da estrutura do banco)
% Também serve para migração e melhoria 

% Colocar como trabalho futuro
% Integração com ORMS (gerador? Quais as linguagens de programação seriam target?) 

%  Não colocar como trabalho Futuro
% Annotations em JAVA - Como usar no Xtext
% Como tratar dados sensíveis? (lei de proteção de dados)
% Usar anotações para estratégia de criação de banco de dados de teste 




