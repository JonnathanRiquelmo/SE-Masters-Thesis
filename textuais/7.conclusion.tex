%=========================================================
\chapter{Final Remarks}
\label{chap:conclusion}
%=========================================================

% Esta dissertação se propôs até o momento a realizar um estudo que aborda a área de projeto e modelagem conceitual de \acp{db} relacionais.
% Para tanto, foi necessário realizar um desenho de pesquisa que orientou toda a execução das atividades necessárias para atingir os objetivos específicos.
% O projeto envolve ainda uma pesquisa bibliográfica para o embasamento teórico, a condução de um mapeamento sistemático, encontrar estudos relacionados, o desenvolvimento da solução proposta e ao menos duas avaliações empíricas para sua avaliação.
This master's thesis has so far proposed to execute a study that addresses the area of design and conceptual modeling of relational \acp{db}.
Therefore, it was necessary to carry out a research design that guided all the activities necessary to achieve the specific objectives.
The project also involves a bibliographical research for the theoretical basis, conducting a systematic mapping, finding related studies, developing the proposed solution and at least two empirical evaluations for its evaluation.

% Sabemos que o conjunto de ferramentas que apoiam o projeto e modelagem de bancos de dados relacionais é relativamente amplo.
% Não esperamos reinventar a roda ou apenas aumentar essa fragmentação composta por diversas soluções em uma área já madura. 
% Em geral, este trabalho trata de uma experimentação.
% Estamos procurando verificar suposições e explorando uma alternativa em relação a abordagem mais utilizada neste meio, e portanto de certa forma desafiando o \textit{status quo} estabelecido.
% Como resultado esperamos contribuir com uma ferramenta que seja potencialmente útil para o processo de ensino-aprendizagem.
We know that the set of tools that support the design and modeling of relational \acp{db} is relatively wide.
We do not expect to reinvent the wheel or just increase this fragmentation made up of several solutions in an already mature area.
In general, this work is about an experimentation.
We are looking to verify assumptions and exploring an alternative to the most used approach in this area, and therefore somewhat challenging the established \textit{status quo}.
As a result we hope to contribute with a tool that is potentially useful for the teaching-learning process.

% Nessa tentativa de expandir os limites e explorar um caminho alternativo, que acreditamos que deva ser encorajado dentro da academia, nós propomos uma \ac{dsl} e implementamos uma ferramenta completa de modelagem.
% Isso só foi possível através do framework Xtext, um conhecido \ac{lw} para criação de linguagens de abordagem textual.
In this attempt to push the boundaries and explore an alternative path, which we believe should be encouraged within academia, we propose a \ac{dsl} and implement a complete modeling tool.
This was only possible through the Xtext framework, a known \ac{lw} for creating textual approach languages.

% Executamos a replicação de um experimento empírico com um conjunto de 33 sujeitos para verificar a viabilidade da proposta e os resultados preliminares indicam que pode haver relevância no projeto construído em nosso estudo.
% Estamos cientes que mesmo que os resultados, após o último experimento planejado a ser realizado, mostrem em última análise que não há diferença significativa da nossa proposta para as ferramentas já existente, ainda assim o nosso conjunto coletivo de conhecimento permanecerá relevante no que diz respeito ao desenvolvimento e evolução de linguagens de modelagem para domínios específicos.
We performed the replication of an empirical experiment with a set of 33 subjects to verify the feasibility of the proposal and the preliminary results indicate that there may be relevance in the project built in our study.
We are aware that even if the results, after the last planned experiment to be carried out, ultimately show that there is no significant difference from our proposal for the existing tools, still our collective body of knowledge will remain relevant with regard to development and evolution of modeling languages for specific domains.

% Temos ainda outras expectativas em relação a ferramenta desenvolvida que vão além de verificar se é uma alternativa viável  e promover colaboração, em razão de ser código aberto, bem como atender às necessidades dos usuários.
% Isto diz respeito a nossa vontade de avaliar se a abordagem textual pode ser de fato tão efetiva quanto a abordagem gráfica. 
% Pretendemos realizar uma melhor análise a partir de uma possível futura inclusão da solução nas atividades realizadas durante a condução de uma classe de banco de dados, assim performando as atividades continuamente da mesma forma com que ferramentas gráficas são utilizadas.
We also have other expectations for the developed tool, which goes beyond checking whether it is a viable alternative and promoting collaboration since it is open-source, as far as to meet the users' needs.
% We also have other expectations to the developed tool that goes beyond checking if it is a viable alternative and promoting collaboration, since it is open-source, as well as meeting the needs of users.
Hence, this concerns our willingness to assess whether the textual approach can be as efficacious as the graphical approach.
% We intend to carry out a better analysis based on a possible future inclusion of the solution in the activities performed while conducting a \ac{db} lessons, thus performing the activities continuously in the same way that graphical tools are used.
We intend to conduct a better analysis based on the possible inclusion of the future solution in the activities carried out while teaching database lessons, thus performing the activities continuously in the same way used in the graphical tools.

% Desta forma será possível reverificar algumas vantagens que observamos inicialmente (Seção \ref{sec:intro_rationale}), e ainda avaliar se outras vantagens não descritas anteriormente podem ser adicionadas neste conjunto de benefícios.
% Entre as suposições iniciais que temos estão a de que a abordagem textual pode trazer grandes vantagem na velocidade de modelagem quando os usuários estão mais acostumados com a linguagem,  pode promever uma integração facilitada com outras linguagens, oferece qualidade superior de formatação, melhor controle de versão e indepêndencia de plataforma (no que diz respeito a sistemas operacionais).

% In this way, it will be possible to recheck some advantages that we observed initially (Section \ref{sec:intro_rationale}), and also to assess whether other advantages not previously described can be added to this set of benefits.
In this way, as we initially observed (Section \ref{sec:intro_rationale}), it will be possible to recheck some advantages and assess whether others not previously described could be added to this benefits set.
% Among the initial assumptions we have are that the textual approach can bring great advantages in modeling speed when users are more familiar with the language, can provide easier integration with other languages, offer superior formatting quality, better version control and platform independence (concerning operating systems).
Among the initial assumptions, we can assert that the textual approach can: bring good advantages in modeling speed when users are more familiar with the language; provide easier integration with other languages; offer superior formatting quality; better version control, and; platform independence (concerning operating systems).

%--------------------------------------------------------------------
\section{Publications}
%--------------------------------------------------------------------

% Além disso, destacamos que durante o período de realização deste estudo os resultados parciais foram submetidos a diversos eventos visando uma avaliação de nossos pares acerca do nosso trabalho.
Furthermore, we emphasize that during the period study, we have submitted the partial results to several events aiming at evaluations of expert peers about our work.
As follows, we list some publications that originated in the context of our research.
% Furthermore, we emphasize that during the period of this study, the partial results were submitted to several events aiming at an evaluation of our peers about our work.
% Below we list some publications that originated in the context of our research.
\linebreak 
\linebreak
\textbf{Published} 
    \begin{enumerate}[label=\roman*.]
        \item Lopes, J., Bernardino, M., Basso, F., Rodrigues. Avaliação preliminar de uma linguagem para a representação textual de modelos conceituais de bancos de dados. In \textit{Anais da IV Escola Regional de Engenharia de Software.} Porto Alegre, RS, Brasil: SBC, 2020. p. 306–315.~\cite{eres}. 
        \item Lopes, J., Bernardino, M., Basso, F., Rodrigues. Multivocal Literature Mapping on DSLs and Tools for Entity-Relationship Modeling. In \textit{Anais da V Escola Regional de Engenharia de Software.} Porto Alegre, RS, Brasil: SBC, 2021.~\cite{eres:2021}
        \item Lopes, J., Bernardino, M., Basso, F., Rodrigues, E. Textual approach for designing database conceptual models: A focus group. In \textit{Proceedings of the 9th International Conference on Model-Driven Engineering and Software Development - MODELSWARD},. [S.l.]: SciTePress, 2021. p. 171–178.~\cite{modelsward21}. 
        \item Lopes, J., Bernardino, M., Basso, F., Rodrigues, E. Empirical evaluation of a textual approach to database design in a modeling tool. In \textit{Proceedings of the 23th International Conference on Enterprise Information Systems (ICEIS)}. [S.l.]: SciTePress, 2021. p. 8.~\cite{iceis21}. 
        \item Lopes, J., Bernardino, M., Basso, F., Rodrigues, E. Textual-based DSL for Conceptual Database Modeling: A Controlled Experiment. In \textit{Anais do XXXVI Simpósio Brasileiro de Bancos de Dados (SBBD)} [S.l.]: SBC, 2021.~\cite{sbbd:2021}
    \end{enumerate} 

% \linebreak
\\~\\
\textbf{Accepted to be Published} 

    \begin{enumerate}[label=\roman*.]
        \item Lopes, J., Bernardino, M., Basso, F., Rodrigues, E. Entity-Relationship Modeling Tools and DSLs: is it still possible to advance the state of the art from observations in practice? In \textit{Proceedings of the 24th International Conference on Enterprise Information Systems (ICEIS)}. [S.l.]: SciTePress, 2022.
    \end{enumerate} 

% \linebreak
\\~\\
\textbf{Under Review}  

\begin{enumerate}[label=\roman*.]
    \item Journal of Information Data Management (JIDM), 2022.
\end{enumerate} 

% \linebreak
% \\~\\
% \textbf{To be Submitted} 
%     \begin{enumerate}[label=\roman*.]
%         \item ---
%     \end{enumerate} 

%--------------------------------------------------------------------
\section{Ongoing Work}
%--------------------------------------------------------------------

% Existem outras melhorias que estamos desenvolvendo mas que não são rigorosamente necessárias para a realização da última avaliação empírica. 
% Pretendemos implementá-las até o final do período previsto para a dissertação, tornando-as públicas no repositório do projeto.
% Estes recursos buscam melhorar a experiência de uso e aumentar o espectro de artefatos que podem ser gerados.
% Entre os principais recursos que pretendemos adcionar, estão:
% \begin{itemize}
%     \item mais templates proposals, especialmente para modelos completos;
%     \item melhorar a validação com base em escopo, em especial para entidades ternárias;
%     \item criar geradores de DDLs para procedimentos armazenados cobrindo operações CRUD;
%     \item continuar a implementação do suporte de mineração de código\footnote {Code Mining: \url{https://www.eclipse.org/eclipse/news/4.8/platform_isv.php}} usando os recursos lançados inicialmente no projeto Eclipse Photon.
%   O objetivo será mostrar resumos ao usuário do número de entidades, incluindo aquelas derivadas de relacionamentos N:N, bem como o total de relacionamentos.
%     \item criar ao menos mais um gerador SQL, aumentando assim o conjunto de plataformas que podem ser alvo.
% \end{itemize}
There are other improvements that we are developing, but that is not strictly necessary to carry out the last experimental evaluation.
We intend to implement them by the end of the period foreseen for the master's thesis, making them public in the project repository.
These features seek to improve the user experience and increase the spectrum of artifacts generation.
Among the main features we intend to add are:

\begin{itemize}
    \item more templates proposals, especially for complete (full) models;
    \item improve scope-based validation, especially for ternary entities;
    \item create DDL generators for stored procedures covering CRUD operations;
    \item to continue implementing code mining\footnote{Code Mining: \url{https://www.eclipse.org/eclipse/news/4.8/platform_isv.php}} support using the features initially released in the Eclipse Photon project.
    The objective will be to show the user summaries of the number of entities, including those derived from N:N relationships, as well as the total of relationships;
    %\item create at least one more \ac{sql} generator, thus increasing the set of platforms that can be targeted.
    \item create at least one more \ac{sql} generator, thus increasing our set of target platforms.
\end{itemize}

\rowcolors{1}{gray!15}{white}
\begin{table}[!htb]
\centering
\footnotesize
\begin{tabular}{lcccccc}
\rowcolor[HTML]{C0C0C0}
\bottomrule
 & \textbf{ERtext} & \textbf{MIST} & \textbf{dbdiagram.io} & \textbf{QuickDBD} & \textbf{bigER} \\ \hline
Web-based &  &  & \checkmark & \checkmark & \\
IDE Integration & Eclipse & Eclipse &  &  & VS Code \\
Open-Source & \checkmark & \checkmark &  &  & \checkmark \\
Free of Charge & \checkmark & \checkmark & partially & partially & \checkmark \\
Textual Editor & \checkmark & \checkmark & \checkmark & \checkmark & \checkmark \\
LSP Implementation & \checkmark & \checkmark &  &  & \checkmark \\
Diagram View & \checkmark &  & \checkmark & \checkmark & \checkmark \\
SQL Generation & \checkmark &  & \checkmark & \checkmark & \checkmark \\
Model Validation & \checkmark &  &  &  & \checkmark \\
Experimental Evaluation & \checkmark & \checkmark &  &  & \\
\toprule
\end{tabular}
\fonte{Author.}
\end{table}